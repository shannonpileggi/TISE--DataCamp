\documentclass{tise_style}
\usepackage{natbib}




\title{Enter our title here}

\author{Tessa Mcdonnel, Shannon Pileggi,  \\Statistics Department, California Polytechnic State University, San Luis Obispo}
%\author{Shannon Pileggi \\Statistics Department, California Polytechnic State University, San Luis Obispo}


\Abstract{ The text of your abstract.}

\Keywords{one, two, three}



\begin{document}



\section{Introduction}

It is no suprise that technology plays a vital role in practicing statistics and as the technological revolution continues to 
unfold, the use of statistical software has become increasingly emphasized in statistics education. By integrating the 
capabilities of technology in introductory statistics courses, students can explore statistical ideas by analyzing and 
interpreting real data sets (Chance2007), giving them an authentic view of how statistics is practiced outside of the 
classroom (Broaddus2013). The revised 2016 \textit{Guidelines for Assessment and Instruction in Statistics Education} (GAISE) 
College Report has also prompted the use of more technology in introductory statistics courses, encouraging educators to use technology "as a way to explore conceptual ideas and enhance student learning" (AmericanStatisticalAssociation2016).



The goal for any introductory statistics course is to help students develope the ability to think statistically (AmericanStatisticalAssociation2016). 
Introducing technology into the curriculum can greatly strengthen this ability by illustrating statistical concepts such as variability, randomness and simulation.

Although numerous statistical technologies exist, \textit{R} and \textit{R Studio} provide extensive statistical capabilites (free of charge) and are commonly used in statistics courses (Chance2007). R allows students to explore real-world data sets which can not only enhance their understanding of concepts, but can also spark an interest in the domain of statistics.


With this increased use of R in classrooms, there is a need for a more effective way to get students familiar with R.
For students who have never been exposed to programing, the \textit{R} interface can be a intimidating. It is often the case
that the main source of student difficulties are with software and not the statistical concepts themselves (Hare2017). To 
overcome this learning curve, instructors utilize R teaching tools such as Swirl and TryR. Swirl is sim



For a statistical learning tool to be effective, the software system should be modular, extensible, reproducable, and web-based (Hare2017). 

Modulariy is critical component in designing a statistical learning tool; it allows independent pieces to be easily added, modified or removed from the entire functionality (Hare2017). 


Check this out, Schubert said this \cite{Schubert13}.


Check this out, Campbell said this \cite{Campbell02} regularly, but cite this in parentheses \citep{Campbell02}.


Check this out, Chi said this \cite{Chi81}.


\bibliographystyle{ECA_jasa}
\bibliography{bibfile}


\end{document} 
