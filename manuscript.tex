\documentclass{tise_style}
\usepackage{natbib}




\title{Enter our title here}

\author{Tessa Mcdonnel, Shannon Pileggi,  \\Statistics Department, California Polytechnic State University, San Luis Obispo}
%\author{Shannon Pileggi \\Statistics Department, California Polytechnic State University, San Luis Obispo}


\Abstract{ The text of your abstract.}

\Keywords{one, two, three}



\begin{document}



\section{Introduction}

It is no suprise that technology plays a vital role in practicing statistics and as the technological revolution continues to 
unfold, the use of statistical software has become increasingly emphasized in statistics education. By integrating the 
capabilities of technology in introductory statistics courses, students can explore statistical ideas by analyzing and 
interpreting real data sets \citep{Chance2007}, giving them an authentic view of how statistics is practiced outside of the 
classroom \citep{Wang2017}. The revised 2016 \textit{Guidelines for Assessment and Instruction in Statistics Education} 
(GAISE) College Report has also prompted the use of more technology in introductory statistics courses, encouraging educators 
to use technology "as a way to explore conceptual ideas and enhance student learning" (AmericanStatisticalAssociation2016).

The goal for any introductory statistics course is to help students develope the ability to think statistically 
\citep{AmericanStatisticalAssociation2016}. 
Introducing technology into the curriculum can greatly strengthen this ability by illustrating statistical concepts such as
variability, randomness and simulation. Although numerous statistical technologies exist, \textit{R} and \textit{R Studio} 
provide extensive statistical capabilites (free of charge) and are commonly used in statistics courses \citep{Chance2007}. R 
allows students to explore real-world data sets which can not only enhance their understanding of concepts, but can also spark
an interest in the domain of statistics \citep{Wang2017}.

With this increased use of R in classrooms, there is a need for a more effective way to get students familiar with R. While R
software provides users with extraordinary abilities there is no doubt a steep learning curve. To facilitate this transition,
popular interactive teaching technologies have been created including Swirl (need to cite this?), LearnR and TryR. These tools 
allow students to work through coding problems while recieving immediate feedback in and out of the classroom. Swirl is an open source R software package 
that users access in their R console. This interactive learning tool contains several pre-existing lessons but what makes 
Swirl so popular is its versitality; educators can create their own lesson (or tailor a pre-existing one) to fit their 
curriculum and assign it to students \citep{Carchedi2014}. 

*I'm confused about LearnR and thinks someone else should write about it.*

TryR (Code School 2014) is a web-based programs and does not have this reproducible capability. Since public content authoring is not available, educators are limited on what they can teach their students. The R console can be intimidating to introductory statistics students so an advantage of web-based teaching tools is 
the clean and easy to use interface. This is an important quality given that a lot of the time students struggle more with 
software than the statistical concepts themselves \citep{Hare2017}. When students only need their web browser to participate,
all installation complications are avoided and valuable class time can be focused on the statistics, not computer issues.


DataCamp is also a web application that offers interactive R programing courses but unlike TryR, educators can 
create their own courses that coincide with the lessons or assign a pre-existing DataCamp course. The DataCamp website offers 
free community courses as well as more specialized classes that you need to purchase. ... (i'll come back to this)

For a statistical learning tool to be effective, the software system should be modular, extensible, reproducable, and web-
based (Hare2017). 

Modulariy is critical component in designing a statistical learning tool; it allows independent pieces to be easily added, 
modified or removed from the entire functionality (Hare2017). 

\section{How to Use DataCamp}
Pre-existing courses:
DataCamp has a range of available free community courses that educators can utilize for supplimental R practice.


Creating a DataCamp Course:
To create a course on DataCamp you will first need to create a DataCamp account by visiting website\www.datacamp.com/teach and
have a GitHub account. The simplest way to begin building the course is to use the templete course files from the 'Create
DataCamp Course' dialog. This will automatically create a GitHub repository that is linked to DataCamp; any changes that are 
make through DataCamp will update the GitHub repository and visa versa. The GitHub repository will contain three files: 
README.md, course.yml, and chapter1.Rmd.
Readme.md contains helpful information of how to get started, citing resourses to learn more about the process. 
course.yml is a file that contains general information about the course you're creating: title, university, description, 
difficulty level, ect.  
chapter1.Rmd is a template chapter file with examples of 'normal' interative coding exercises and multiple choice questions.
To add more chapters to an R course, create new files in the repository and name them chapter1.Rmd, chapter2.Rmd, chapter3.Rmd
ect.

DataCamp courses can contain 3 different types of exercises: Normal, Multiple choice, and video.


Check this out, Schubert said this \cite{Schubert13}.


Check this out, Campbell said this \cite{Campbell02} regularly, but cite this in parentheses \citep{Campbell02}.


Check this out, Chi said this \cite{Chi81}.


\bibliographystyle{ECA_jasa}
\bibliography{bibfile}


\end{document} 
