\documentclass{beamer}
\usepackage{graphicx}
\usepackage[utf8]{inputenc}
\usepackage{hyperref}
\hypersetup{colorlinks=true, linkcolor=black}
\usepackage{url}
%Information to be included in the title page:
\title{Creating DataCamp courses}
\subtitle{An interactive teaching tool}
\author{Tessa McDonnel}
\institute{Cal Poly San Luis Obispo}
\date{}
 
 
 
\begin{document}
 
\frame{\titlepage}

\begin{frame}
\frametitle{questions for Pileggi}


1) How much detail should I go into when I show an exercise? Should I say anything about the data set? Should I list all of the parts of a lesson and give a little info on each? I'm definitely going to demonstrate the interactive-ness on it by making errors and showing how different error messages come up.

2) Should I say `R coding' or `R programming' or just `R'

3) How do I add 2 links that have space in between? I want to add the link to the teach editor and a link to the course page if the first link doesnt work

4) Can I change the success message to something more intersting? Or maybe go on and show them the next question which is multiple choice about the assumtions?

5) I'm rethinking the demo exercise because it is kind of long. Maybe I should use lab 3 exericse 9 (summarizing categorical data) because there is just one instruction.
\end{frame}

\begin{frame}
\frametitle{Purpose of reasearch project}

This summer I worked with Dr. Pileggi to improve upon the STAT 217 labratory assignments in R by creating
a supplimental online DataCamp course that accompanies the labs. 

\begin{itemize}
 \item<1-> Technology plays a huge role in statistics education.
 \item<2-> Non-statistics major students have a lot of trouble with statistical software.
 \item<3-> The previous STAT 217 pre-labs consisted of a lab manual and video tutorial.
 \item<4-> My job this summer was to convert the previous lab materials to an interactive online R programming course on DataCamp.
\end{itemize}

\end{frame}
 
\begin{frame}
\frametitle{What is DataCamp?}
\begin{itemize}
 \item<1-> interactive leaning tool; receive instant personalized feedback
 \item<2-> educators can create their own DataCamp courses or utilize the free existing courses 
 \item<3-> DataCamp is a web-based application, only need a web browser to access courses (no R installation required)
 \item<4-> Free for academic groups 
 \item<5-> DataCamp can be linked to learning management systems, easy to track student progress
\end{itemize}
\end{frame}

\begin{frame}
\frametitle{Creating DataCamp Course}
\textbf{Step 1: Review}
For each of the 8 labs, I took notes of:
\begin{itemize}
 \item Statistical concepts used
 \item R commands used 
 \item What kind of data set and variable types were used
 \item[]
\end{itemize}
\pause
\textbf{Step 2: Practice}
To get a better understanding of DataCamp courses:
\begin{itemize}
  \item I completed a few existing courses on DataCamp
  \item[]
\end{itemize}
\pause
\textbf{Step 3: Create}
To convert the existing labs into an online course:
\begin{itemize}
  \item read through all of the DataCamp documentation
  \item started building the course
\end{itemize}
\end{frame}

\begin{frame}
\frametitle{Demonstration}
Lab 5 Exercise 2: Assumptions of the t-test

\centering
\href{https://www.datacamp.com/teach/edit/repositories/1294/branches/master}{DataCamp Edit Page}

\centering
\href{https://www.datacamp.com/courses/stat-217-r-course}{The STAT 217 Course Page}
\end{frame}
\begin{frame}
\begin{figure}[h]
  \includegraphics[scale = 0.28] {Exercise_edit.jpg}
\end{figure}
\end{frame}

\begin{frame}
\begin{figure}[h]
  \includegraphics[scale = 0.25] {exercise_preview.jpg}
\end{figure}
\end{frame}
\begin{frame}
\frametitle{Tools}
\begin{block}{Tools Used for Learning}
PolyLearn 

DataCamp
\end{block}
\begin{block}{Tools Used for creating course}
GitHub

DataCamp
\end{block}
\begin{block}{Tools Used For Paper}
Mendeley

LaTeX
\end{block}
\end{frame}
\begin{frame}
\frametitle{Acknowledgements}
Bill and Linda Frost Fund 

Frost Undergraduate Student Research Award
\end{frame}
\end{document}