\documentclass{beamer}
 
\usepackage[utf8]{inputenc}
\usepackage{hyperref} 
\usepackage{url}
%Information to be included in the title page:
\title{Creating DataCamp courses}
\subtitle{An interactive teaching tool}
\author{Tessa McDonnel}
\institute{Cal Poly San Luis Obispo}
\date{}
 
 
 
\begin{document}
 
\frame{\titlepage}

\begin{frame}
\frametitle{questions for Pileggi}

1) For the slide `Creating DataCamp Course', how can I show the Step 1 bullets first, then the step 2?
Right now it is showing the first point in both, then second, ect.

2) I'm looking through the DataCamp course and finding a good exercise to show in the presentation but I'm seeing a lot of 
errors... The subscripts for hypotheses are not working anymore (at least when I view the course).

3) Should I say `R coding' or `R programming' or just `R'
\end{frame}

\begin{frame}
\frametitle{Purpose of reasearch project}

This summer I worked with Dr. Pileggi and improved upon the STAT 217 labs in R by creating
a supplimental online DataCamp course that accompanies the labs. 

\begin{itemize}
 \item<1-> Technology plays a huge role in statistics education.
 \item<2-> Non-statistics major students have a lot of trouble with statistical software.
 \item<3-> The previous STAT 217 pre-labs consisted of a lab manual and video tutorial.
 \item<4-> My job this summer was to convert the previous lab materials to an interactive online R programming course.
\end{itemize}

\end{frame}
 
\begin{frame}
\frametitle{What is DataCamp?}
\begin{itemize}
 \item<1-> interactive leaning tool; receive instant personalized feedback
 \item<2-> Similar to swirl, educators can create their own DataCamp courses or utilize the free existing courses 
 \item<3-> DataCamp is a web-based application, only need a web browser to access courses (no R installation required)- can be very helpful for intro stats 
 \item<4-> Free for academic groups 
 \item<5-> DataCamp can be linked to learning management systems, easy to track student progress
\end{itemize}
\end{frame}

\begin{frame}
\frametitle{Creating DataCamp Course}
\textbf{Step 1: Review}
For each of the 8 labs, I took notes of:
\begin{itemize}
 \item<1-> Statistical concepts used
 \item<2-> R commands used 
 \item<3-> What kind of data set and variable types were used
\end{itemize}
\textbf{Step 2: Create}
To convert the existing labs into an online course:
\begin{itemize}
  \item<1-> completed a few existing courses on DataCamp 
  \item<2-> read through all of the DataCamp documentation
  \item<3-> started building the course
\end{itemize}
\end{frame}

\begin{frame}
Exercise code and preview mode from a Lab: demonstrate the interactive ability of DataCamp. (thinking maybe the arithemic slides? I want to use a later lab but I'm not happy with any of the exercises so I'd want to edit them and add more in depth error messages.)

\textbf{Insert url to the DataCamp course page}

\end{frame}

\begin{frame}
\frametitle{New Tools}
\begin{block}{Tools Used for Learning}
PolyLearn 

DataCamp
\end{block}
\begin{block}{Tools Used for creating course}
GitHub

DataCamp
\end{block}
\begin{block}{Tools Used For Paper}
Mendeley

LaTeX
\end{block}
\end{frame}
\end{document}